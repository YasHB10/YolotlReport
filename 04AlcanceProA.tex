\chapter{Alcance del proyecto}
	\section{Objetivos del proyecto} \label{Sec_ObjetivosPro}
	En esta sección se habla de los objetivos, tanto generales cómo específicos, 
	que persigue el presente trabajo terminal.
		\subsection{Objetivos generales}\label{Sec_ObjetivosGen}		
			\begin{itemize}
				\item Fomentar la cultura Mexica entre jóvenes mayores de 13 años a
				 través de un videojuego.
			\end{itemize}
			
		\subsection{Objetivos especificos} \label{Sec_ObjetivosEsp}
			\begin{itemize}
				\item Realizar una investigación sobre la cultura Mexica.
				\item Diseñar un videojuego con bases históricas y mitológicas.
				\item Diseñar e implementar una narrativa que permita la difusión de 
				la cultura Mexica. 
				\item Comprender el funcionamiento del motor de juego elegido.
				\item Entender el funcionamiento de un juego de plataforma básico.
			\end{itemize}
			
	\section{Alcance del proyecto} \label{Sec_Alcance}
		El presente trabajo terminal tendrá:
			\begin{itemize}
				\item Funcionalidad de un solo usuario.
				\item Contener diez niveles, uno introductorio y nueve situados en el
				 inframundo Mexica.
				\item Contar con sprites originales.
				\item Contar con un sistema de guardado, para salvar el progreso del 
				jugador.
				\item Contener cinematicas que cuentan una historia original.
				\item Funcionar en dispositivos Android con los requerimientos expuestos 
				en la sección \ref{Sec_Plataforma}.
				\item Contener un nivel que permite repetir los niveles ya completados. 
			\end{itemize}
		El presente trabajo terminal no realizará:
		\begin{itemize}
			\item Enseñar historia.
			\item Realizar microtransacciones.
			\item Soportar múltiples jugadores.
			\item Contar con música original, creada especialmente para el juego.
		\end{itemize}
	\section{Metodologia de trabajo}\label{Sec_Metodologia}
	La metodología de trabajo elegida es Huddle. Como se mencionó en el apartado
	 \ref{MetodoVideojuego}, Huddle es una metodología orientada a videojuegos y una
	  de sus principales bondades que que ofrece la naturaleza iterativa de Scrum 
	  con el agregado de cubrir la linea de producción de un videojuego 
	  (ver apartado \ref{Pipelinevideojuego}).
	\\
	\par
	El principal motivo por el que se eligió huddle, fue que es una metodología 
	orientada a videojuegos; por lo que su documentación y sistema de trabajo cubre 
	las necesidades de un proyecto de esta naturaleza y no es necesario hacer 
	adaptaciones drásticas de la metodología tal como se tendrían que hacer si se 
	hubiera elegido alguna de las metodologías orientadas a desarrollo de software 
	como hubiera sido Scrum o programación extrema. 
	\\
	\par
	Para consultar el cronograma de actividades del Trabajo Terminal, consultar el anexo \ref{}.
	
	\section{Especificaciones de plataforma}\label{Sec_Plataforma}
	En esta sección se listarán todos aquellos dispositivos de hardware y licencias
	 de software que se necesitan para el desarollo del videojuego.
	 
	 \subsection{Hardaware requerido}\label{Sec_PlataformaHw}
	 En esta sección se mencionan los dispositivos de hardware empleados en el 
	 desarrollo del sistema y los dispositivos de prueba del juegos. Estos 
	 dispositivos son con los que se contaban a la hora de iniciar el Trabajo Terminal
	 y no son sustituibles por motivos de presupuesto.
	 	%================================================
	 	\subsubsection{Computadoras para desarrollo}
	 		\begin{itemize}
	 			\item Computadora DELL Inspiron 15.
	 				\begin{itemize}
	 					\item Procesador Intel Core i3-4005U. 
	 					\item CPU de 1.70 GHz de 64 bits. 
	 					\item Memoria ram de 8GB.
	 				\end{itemize}
				\item Lenovo G40. 
					\begin{itemize}
						\item Intel Core i3 4005U CPU 1.7 Khz de 64 bits. 
						\item Memoria ram de 8GB. 
						\item Tarjeta gráfica AMD Radeon R5 235 de 1GB
					\end{itemize}
	 		\end{itemize}
	 		
	 	%===================================
	 	\subsubsection{Dispositivos moviles de prueba}
	 		\begin{itemize}   		
				\item Dispositivo de prueba 1:
			   		\begin{itemize}
			   			\item Versión 5.2
			   			\item Modelo Huawei TAG-L13
			   			\item CPU MediaTek MT6753 1,50 GHz
			   			\item IPS TFT 16M colors 720 x 1280 px (5,00) 294 ppi
			   			\item RAM 2GB	   			
			   		\end{itemize}
	   		
	   			\item Dispositivo de prueba 2:
			   		\begin{itemize}
			   			\item Versión 7.0
			   			\item Modelo ASUS X008DC
			   			\item CPU MediaTek Quad Core Processor
			   			\item GPU Mali T720
			   			\item RAM 3GB LPDDR3
			   			\item PANEL 5.2-inch
			   			HD(1280 x 720) IPS display 
			   			75 por ciento screen-to-body ratio
			   			400nits brightness 
			   		\end{itemize}   		  		
	   		\end{itemize}
	
	
	\subsection{Software requerido}
		\subsubsection{Motor de juego}
	Como motor de juego se optó por Unity 3D en su versión 5.6.2.f1 como 
	motor de juego en su licencia libre ya que no se cuenta con los fondos necesarios 
	para contratar las versiones de pago. Los motivos por los que se eligió Unity 3D,
	 son los que se presentan a continuación:
	        \begin{itemize}
				\item Curva de aprendizaje rápida.
				\item Comunidad de desarrolladores activa.
				\item Permite gestionar trabajos 2D y 3D, esto permitirá escalar el
				 juego a 3D a futuro si alguien deseara retomar el proyecto.
				\item Codificación basada en el paradigma de programación orientada a
				objetos.
				\item Requerimientos técnicos de instalación dependientes del proyecto 
				por lo que no exige una computadora de gran costo.
				\item Capacidad de desarrollo en múltiples plataformas, lo que permite 
				la escalabilidad futura del proyecto hacia nuevas plataformas en caso de 
				que alguien desee retomarlo.
	        \end{itemize} 
	A fin de garantizar la generación de los archivos apk de juego fue necesaria la
	 instalación de Android Studio versión 2.3.3 y java en su versión  8u111.
		%================================
		\subsubsection{Creación de sprites}
	Para la creación de sprites se eligieron dos softwares Corel Draw X5 y Adobe 
	Photoshop. El primero se eligió para la vectorización de los sprites, ya que es de 
	fácil uso, no requiere tantos recursos como Adobe Ilustrator y permite importar 
	archivos a Adobe Photoshop para su posterior coloreado.
	\\
	\par
	Tal como se mencionó en el párrafo anterior, el objetivo de Adobe Photoshop 
	dentro de este trabajo terminal es colorear los sprites. El motivo para emplear 
	Adobe Photoshop es que permite la edición de imágenes y su optimizado para hacer 
	sprites que requieran menores tiempos de renderizado y menor espacio de 
	almacenamiento sin sacrificar significativamente la calidad de la imagen.
		%================================	
		\subsubsection{Interfaz gráfica de usuario}
	Para los iconos de los botones que controlan al usuario, se descargó una colección 
	de botones del sitio web pixelsticky, este sitio web permite la descarga y 
	utilización de diferentes iconos bajo la licencia de CC0 o de dominio público.
	
	\section{Productos esperados}\label{Sec_Producto}
	Los productos esperados se dividirán en dos, siendo los primeros los que se 
	entregarán al termino de Trabajo Terminal 1 y los segundos los que se entregarán
	 al finalizar Trabajo Terminal 2.
	
	\begin{itemize}
		\item Productos esperados al finalizar TT1:
			\begin{itemize}
				\item Documento de diseño.
				\item Guión literario del juego.
				\item Storyboard del juego.
				\item Documentación de la propuesta de diseño del juego.
				\item Maquetado de los niveles 1, 2, 3, 4.
				\item Niveles 1 y 2 terminado.
				\item Reporte técnico.
			\end{itemize}
		\item Productos esperados al finalizar TT2:
			\begin{itemize}
				\item Documentación del juego actualizada.
				\item Maquetado de los niveles 5, 6, 7, 8, 9 y 10.
				\item Nivel 3, 4, 5, 6, 7, 8, 9 y 10 terminados.
				\item Reporte técnico.
			\end{itemize}
	\end{itemize}