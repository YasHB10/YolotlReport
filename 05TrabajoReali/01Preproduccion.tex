\section{Etapa de Preproducción}
En esta sección se hablarán a manera de resumen de la etapa de preproducción 
desarrollada durante Trabajo Terminal 1.

\subsection{Primer Sprint Huddle de Preproducción}
Antes de iniciar el diseño del juego se realizó un trabajo de investigación 
sobre la cultura azteca. Esta investigación abarcó:

\begin{itemize}
	\item \textbf{La sociedad Mexica:} su historia tradiciones y clases sociales. 
	\item \textbf{Mitología mexica:} Dioses, mito de los cinco soles, mito de la 
	creación del hombre del maíz, el Mictlán.
	\item \textbf{Historia de la Malinche:} Historia del personaje antes y después 
	de la llegada de los españoles.
\end{itemize} 
 
Durante la etapa de investigación se seleccionó la información histórica que 
sería relevante y útil para la narrativa del juego y el diseño de su jugabilidad. 
Para la investigación histórica de esta etapa se consultaron libros, códices, 
páginas de internet, artículos de investigación e incluso se visitaron museos 
como el templo mayor.

\subsection{Segundo Sprint Huddle de Preproducción}
En esta etapa se redactaron las primeras secciones del documento de diseño del 
juego Yolotl. Primeramente, se inició con la idea concepto del juego. Para algunos 
juegos la mecánica es la primera es ser definida; no obstante, por la naturaleza 
del juego como herramienta de transmisión de cultura, Yolotl nacio con su historia. 
La historia de Yolotl paso por deferentes etapas de diseño en el que se vio 
modificada, pero manteniendo algunos elementos y agregando otros. 
\\
\par
En la etapa del concepto también se definieron la plataforma para la que sería 
el juego: dispositivos móviles con sistema operativo Android 5.2; mientras que 
la plataforma de desarrollo sería el motor de juego Unity3D. Paralelamente a la 
preproducción se inicia el desarrollo de un primer demo que permita familiarizarse 
con la herramienta de Unity3D.
\\
\par
Una vez teniendo la idea concepto se definió la visión del juego y sus mecánicas. 
En cuestión de las mecánicas el enfoque por el que se optó fue el de mantener 
el juego con mecánicas simples y familiares para aquellos jugadores que ya habían 
tenido alguna experiencia con algún juego de plataformas, sin descartar algunos 
detalles que le dieran identidad al juego en cuanto a su jugabilidad. 
\\
\par
Con la historia, la visión y la mecánica definidas se procedió a definir los 
estados del juego, diseñar las interfaces graficas de navegación y de interacción 
con el personaje. Para ver las interfaces ver anexo \ref{Anexo:Intefaces}.

\subsection{Tercer Sprint Huddle de Preproducción}
En el tercer Sprint se definieron la cantidad de niveles y en que consistiría 
cada uno, de igual forma se definieron los objetivos de cada nivel, la recompensa 
a obtener una vez, los enemigos a vencer, las cinemáticas que fungirían como 
transiciones entre nivel.  
\\
\par

Al mismo tiempo que se definieron los niveles se definieron y detallaron los 
personajes tanto a nivel narrativo como a nivel de jugabilidad, definiendo habilidades 
para los enemigos, los niveles en los que parecerían y sus acciones dentro de 
la historia. Para esta parte se trató de obtener la mayor fidelidad posible a 
los mitos y códices. Si se desea profundizar en los diseños de personajes ver el 
anexo \ref{Anexo:Personajes}.

\subsection{Cuarto Sprint Huddle de Preproducción} 
En el cuarto sprint se terminó de definir el argumento del juego al escribir el 
guión de la historia. En este sprint también se definieron elementos de ambientación 
para el juego tales como la música de fondo, los efectos de sonido, los efectos 
de sonido. 
\\
\par
De igual forma, en este sprint se definieron las armas de los personajes, los 
ítems; estos elementos se definieron tanto a nivel de comportamiento como a nivel 
visual. Al igual que con los personajes se buscó que las armas, tanto en 
comportamiento como en diseño, se mantuvieran lo más fiel posible a los mitos 
y leyendas de donde se basaron.
\\
\par
Con el cuarto sprint se finalizó la etapa de preproducción, obteniendo así un 
documento de diseño lo suficientemente detallado como para iniciar el diseño 
del juego a nivel de ingeniera.