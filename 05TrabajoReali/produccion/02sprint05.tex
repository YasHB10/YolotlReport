\subsection{Quinto Sprint Huddle de Producción}
En este sprint se trabajaron sobre las observaciones realizadas por los sinodales 
durante la exposición del Trabajo Terminal 1. 

\subsubsection{Modelo de datos}
La primera observación en atender fue el modelo de datos del juego, dicho modelo 
de datos se realizó utilizando un modelo entidad relación de base de datos (Ver 
Anexo \ref{Anexo:ModeloDatos}) ya que al modelarse de esta forma hace escalable 
el juego si se deseará en algún futuro emplear una base de datos para mejorar 
el almacenamiento de datos y el manejo de más usuarios para ofrecer un modo 
online. El modelo de datos está basado en el modelo de clases y contiene 
únicamente a las clases actoras. Toda entidad actora se define como una 
especialización de una entidad base llamada GameObject, esta entidad está 
definida por como su identificador y por otras entidades como GameObjectPosition, 
Level, Tag, AnimationMachine, entre otros. 

\subsubsection{Control de adicción en el jugador}
La segunda observación sobre la que se trabajo fue como disminuir la adicción 
del jugador al videojuego Yolotl. Esta observación dio lugar a una investigación 
sobre la adicción a los videojuegos ya que antes de proponer alguna solución se 
debía conocer cómo se definía, las causas y las consecuencias de la adicción al 
videojuego. Al final de la investigación se pudieron formular tres posibles 
soluciones para evitar la adicción del jugador; sin embargo, dado que este tópico 
no estaba en la planeación original del proyecto y por las implicaciones que 
conllevaban cada una de las soluciones se decidió únicamente describir las 
soluciones y sus implicaciones sin desarrollar ninguna de las tres. A continuación, 
se describen a manera de resumen las soluciones (nuevamente si se dese a 
profundizar en la investigación realizada y las soluciones se puede consultar 
el Anexo \ref{Anexo:AdiccJuga}):
	\begin{itemize}
		\item \textbf{Notificación de confirmación para continuar la partida.} Esta 
		solución propone que el juego solicite la confirmación del usuario para 
		continuar una vez que éste ha detectado que el jugador ha estado jugando 
		durante un tiempo prolongado como una hora.
		\item \textbf{Control paterno.} El juego le envía un formulario al tutor del 
		jugador por medio de un correo electrónico. En este formulario el tutor podrá 
		decidir cuanto tiempo al día la aplicación podrá estar abierta. 
		\item \textbf{Sistema de vidas.} El jugador tiene una cantidad de vidas 
		limitadas. Cada vez que el jugador ingresa a un nivel o muere dentro de 
		uno y reinicia la partida se gasta una vida. Para recuperar vidas el 
		jugador deberá de esperar un determinado tiempo.
	\end{itemize}
\subsubsection{Modelo de negocios}
Una vez terminados los puntos anteriores se procedió a diseñar el posible plan 
de negocios que se emplearía para el desarrollo y comercialización del juego. 
(Falta integrar el plan de negocios)
	