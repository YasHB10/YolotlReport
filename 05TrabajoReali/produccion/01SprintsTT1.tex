\subsection{Primer Sprint Huddle de Producción.}
En este sprint se realizó un análisis del documento de diseño y se definieron las 
clases y el modelo bajo el que funcionaría el juego a nivel de programación. 
Haciendo uso del paradigma orientado a objetos se propone emplear tres tipos de clases:

\begin{itemize}
	\item \textbf{Actores:} Son las clases que modelan a los enemigos, los ítems, 
	los coleccionables, los checkpoints y al jugador.
	\item \textbf{Controladores:} Son las clases encargadas de gestionar la partida 
	y la navegación entre interfaces. Estas clases desencadenan eventos conforme a 
	las acciones de las clases actoras. Estas clases también son las encargadas de 
	verificar que se cumplan las reglas de los niveles.
	\item \textbf{Auxiliares:} Estas clases ayudan al funcionamiento de los actores 
	y los controladores. Estas clases también se encargan le vincular datos con 
	las clases controladoras como efectos de sonido, música, datos para la 
	progresión entre niveles.
\end{itemize}

El modelo planteado permitió reutilizar parte del demo generado durante la etapa 
de preproducción. 
\\
\par
En el primer Sprint de Producción también se crean los sprites del primer nivel utilizando la herramienta de modelado en 3D Blender.

\subsection{Segundo Sprint Huddle de Producción.}
En este sprint se maqueta la sección del mercado del primer nivel del juego. Y 
se procede a trabajar en los sprites. Durante este sprint se inicia la 
integración del código del primer prototipo con el nuevo modelo del juego. Al 
finalizar este sprint se determina la no viabilidad del modelado en 3D de los 
sprites por cuestiones de tiempos; en consecuencia, se descarta este método para 
generar los sprites y se inicia el desarrollo de los sprites a partir de otras 
técnicas de animación más tradicionales.

\subsection{Tercer Sprint Huddle de Producción.}
En este sprint se inicia el desarrollo de los sprites con Adobe Photoshop y Corel 
Draw. A la par se inicia la maquetación de la etapa de selva del nivel uno. Una 
vez terminados los sprites referentes al nivel uno estos se integran al código 
permitiendo tener un segundo prototipo con la siguiente funcionalidad:

\begin{itemize}
	\item Control de personaje por medio de la GUI.
	\item Transiciones entre interfaces.
	\item Personaje seguible que aparece en el primer nivel funcional.
	\item Funcionamiento básico del controlador de diálogos.
\end{itemize}

\subsection{Cuarto Sprint Huddle de Producción.}
Durante el cuarto sprint se desarrollaron los sprites referentes a los obtaculos 
de plataformas, objetos de fondo. Procediendo a redactar el reporte técnico 
referente al trabajo terminal 1 y la preparación de la posterior exposición.