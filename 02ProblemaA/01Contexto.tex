\section{Contexto}
En México existe un problema de deficiencia en el sistema educativo, así lo
señalan diferentes pruebas cuya finalidad son medir y comparar el nivel de 
conocimientos de los estudiantes mexicanos entre entidades estatales dentro del 
territorio nacional y con respecto a otros países. El Programa para la Evaluación
 Internacional de Estudiantes (P.I.S.A.) es una de las pruebas encargadas de 
medir el desempeño del sistema educativo de un país. Esta prueba es realizada por 
la Organización para la Cooperación y el Desarrollo Económico (O.C.D.E.). La O.C.D.E 
esta compuesta por 35 países, entre ellos México. La prueba P.I.S.A. tiene como 
objetivo evaluar hasta que punto los estudiantes cercanos a concluir la educación 
obligatoria han desarrollado los conocimientos y habilidades necesarios para la 
participación plena en la sociedad del saber\cite{RefOCDE}, por lo que se realiza en 
jóvenes de 15 años en 72 países. En el 2016, México ocupó lugares 58, 55 y 56 de
 en la prueba PISA en materia de conocimiento científico, lectura y comprensión 
lectora y matemáticas respectivamente\cite{RefPisa}, lo que ubica el nivel de 
aprendizaje en México por debajo de la media internacional.
        \\
        \par
Anudado a las deficiencias del sistema educativo, México presenta un 
déficit en cuanto a libros libros leídos, esto se puede decir con base 
en diferentes estudios que determinaron la cantidad de libros leídos en 
México, de los cuales tres son los más citados por especialistas en el 
tema, éstos son: 
	\begin{itemize}
		\item El realizado por la Organización de las Naciones Unidas (O.N.U.), en 
		donde México tuvo una media de 2.8 libros leídos anualmente, ubicando a México
		 en el penúltimo lugar de entre 108 países\cite{RefLibrosCantidad}.
        \item  La Encuesta Nacional de Lectura y Escritura de Conaculta 
        en donde se obtuvo un promedio 5.3 libros al año, entre mexicanos 
        mayores a 13 años\cite{RefLibrosCantidad}.
        \item El más reciente Modulo de Lecutura (M.O.L.E.C.) del Instituto Nacional 
        de Estadística y Geografía(I.N.E.G.I.). Con base en los resultados obtenidos 
        durante la encuesta del 2018, el 76.4\% de la población mayor de 18 años
         alfabeta lee algún material considerado por el M.O.L.E.C; esta cifra 
         presupone una disminución del 3.3\% con respecto al 2017. De este 76.4\% 
         solo el 45.1\% tiene como material de lectura los libros, de los cuales 
         el 40.8\% prefiere los libros de literatura\cite{RefModuloLectura}.
	\end{itemize}	    

