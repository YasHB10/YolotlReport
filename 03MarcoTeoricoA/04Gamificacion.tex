\section{Gamificación}

\subsection{¿Qué es la gamificación?}
La gamificación es comúnmente definida como el uso de elementos del juego y el 
pensamiento basado en juego en entornos no relacionados con el juego para aumentar 
el compromiso o modificar el comportamiento\cite{RefIntroGamificacion}. Otra 
definición para gamificación es la empleada en el libro Gamification by Desing, 
en la cual la gamificación es “un  proceso 
relacionado con el pensamiento del jugador y las técnicas de juego para atraer 
a los usuarios y resolver problemas”\cite{zichermann2011gamification}. 
\\
\par
Es importante comprender que en primera instancia el principal objetivo de la 
gamificación es influir en el comportamiento de las personas al producir un 
sentimiento de dominio y autonomía.  
\\
\par
Actualmente muchas empresas utilizan principios de gamificación en sus sitios 
web o en sus aplicaciones móviles para mejorar la experiencia de sus usuarios y 
conseguir que estos pasen más tiempo en sus aplicaciones o sitios web. Un 
ejemplo del uso de la gamificación en una aplicación móvil es la aplicación de 
Duolingo, en esta aplicación el jugador pude aprender diferentes idiomas a partir 
de dinamicas que involucran la escritura, lectura y audición del idioma. 

\subsection{Elementos del videojuego en la gamificación.}
 Tal como lo dice su definición la gamificación toma elementos del videojuego 
 para cumplir sus objetivos, en este apartado se detallan tres elementos de 
 la gamificación que son tomados del videojuego:

	\begin{itemize}
		\item \textbf{Dinámicas}: Dentro de los videojuegos las dinámicas hacen 
		referencia a la interacción de los objetivos y acciones del jugador con las 
		reglas del juego, es decir la jugabilidad \cite{RefMecanica}; un ejemplo de 
		dinámicas puede ser un sistema de diálogos en el cual el jugador debe de 
		obtener información para resolver diferentes acertijos.

		\item \textbf{Mecánicas}: Las mecánicas son las reglas del juego, dependen 
		directamente de las dinámicas\cite{RefMecanica}. Retomando el ejemplo anterior 
		del sistema de diálogos, sus respectivas mecánicas serían que se tenga un 
		tiempo limite para elegir las respuestas del dialogo, que se disponga de cuatro 
		opciones de respuestas a cada dialogo, etc.

		\item \textbf{Componentes}: Son los últimos elementos en ser elegidos y 
		tienen que ser coherentes con las dinámicas y las mecánicas o de lo contrario 
		la aplicación no tendrá cohesión. Un ejemplo de elementos: avatares, logros, 
		insignias, batallas contra jefes, colleccionables, contenido desbloqueable, 
		niveles, puntos, equipos, tablas de posiciones, misiones, etc 
		\cite{RefIntroGamificacion}.
	\end{itemize}

\subsection{Tipos de jugadores.}
Para diseñar correctamente una experiencia con gamificación es necesario 
delimitar quienes van a usar el producto, que habilidades tienes, etc. El 
sociólogo Richard Bartle desarrollo cuatro tipos de perfiles de jugadores para 
describir los jugadores del juego World of Warcraft; por su escalabilidad esta 
segmentación de jugadores ha sido ampliamente utilizada para clasificar a los 
jugadores sin importar el juego.
\\
\par
La clasificación de Bartle se basa en un modelo de cuatro jugadores puestos sobre 
dos ejes: 
\begin{itemize}
	\item El eje horizontal que describe si el jugador esta más enfocado en el mundo 
	del juego o en otros jugadores.

	\item El eje vertical que representa la afinidad del jugador por actuar o 
	por interactuar.
\end{itemize}

De estos dos ejes se generan cuatro tipos de jugador:
\begin{itemize}
	\item \textbf{Triunfador}: Son todos aquellos jugadores que prestan mayor 
	atención a su estatus y sus puntos dentro del juego. Les gusta mostrar su 
	progreso a sus amigos. Pueden ser motivados por medallas, por incrementos de 
	nivel (ya sea de su personaje o sus armas) dentro del juego. Se calcula que al 
	menos el 10\% de los jugadores son del tipo Triunfador.
	\item \textbf{Explorador}: Los jugadores que son de este tipo desean explorar 
	todo el mapa y descubrir todos los secretos posibles. Para los exploradores el 
	descubrimiento es la recompensa y su mayor motivación es lo que esta oculto en 
	el mundo del videojuego y no alardear con sus amigos sobre sus descubrimientos. 
	Al igual que el tipo triunfador, el 10\% de los jugadores son de este tipos.
	\item \textbf{Socializador}: Este tipo de jugador encuentra su motivación en la 
	interacción con otros jugadores dentro del juego. El jugador socializador 
	preferirá un juego que le permita colaborar con otros jugadores para superar 
	retos como equipo. La mayoría de los jugadores pertenecen a este tipo.
	\item \textbf{Asesino}: Los jugadores de tipo asesino son parecidos a los de 
	tipo triunfador: disfrutan de obtener la victoria y de progresar dentro del 
	juego; sin embargo lo que los diferencia del tipo triunfador es que ellos 
	disfrutan más de ver a otros perder. Buscan ser los mejores en el juego y para 
	ello deben de derrotar a todos los demás, siendo esa su principal motivación. 
	Menos del 1\% de los jugadores pertenece a este tipo de jugadores
	\cite{RefTipoJugadores}.
\end{itemize}

\subsection{Intervalos de actividades de la gamificación.}
Lograr el compromiso hacia un comportamiento o idea es el principal objetivo de 
la gamificación y para conseguir dicho objetivo son necesarios dos tipos de bucles 
en el diseño de la aplicación:
\begin{itemize}
	\item Bucles de compromiso. 
	\item Bucles de inicio y progresión.
\end{itemize}
En los siguientes apartados se profundizara en dichos bucles con el fin de que 
dichos conceptos queden claros al lector.
	\subsubsection{Bucles de compromiso}
	El bucle de compromiso reafirma el compromiso del jugador para que continué 
	mostrando el comportamiento que se desea crear o modificar, este tipo de bucle 
	ata al jugador al juego una vez que ha entrado. En el bucle de compromiso se 
	encuentran los siguientes tres elementos:
	\begin{itemize}
		\item Motivación.
		\item Acción. 
		\item Retroalimentación\cite{RefIntroGamificacion}.
	\end{itemize}

Estos tres elementos se deben de encontrar siempre presentes, por lo que es 
importante que se repitan de manera indefinida. Estos tres elementos deben de 
ser ejecutados en el orden en el que se mencionan ya que si no hay motivación es 
imposible que el jugador pueda pasar a la acción y después de la acción debe de 
existir la retroalimentación o de lo contrario la motivación no se podrá mantener. 
La implementacion de estos elementos depende directamente del tipo de jugador 
objetivo. Por ejemplo para un jugador de tipo explorador su motivación será 
descubrir los secretos del mundo del juego, por lo que sus acción será explorar el 
mundo del juego y su retroalimentación será descubrir un secreto dentro del juego. 
	\subsubsection{Bucle de inicio y progresión}
Este tipo de bucle tiene como objetivo asegurarse que el jugador pueda progresar. 
Para lograr este el sentimiento de progreso en el jugador se necesitan dos tipos 
de bucles:
\begin{itemize}
	\item \textbf{Bucle de inicio}: Este bucle se presenta la primera vez que el jugador 
	juega por primera vez. Este bucle permite al jugador familiarizarse con todas las 
	mecánicas y conceptos del juego. 
	\\
	\par
	\item \textbf{Bucle de progresión}: En este bucle se presentan retos que superan las 
	habilidades iniciales del jugador por lo que éste deberá de mejorar para poder 
	progresar. Este reto tiene como objetivo medir cuanto el jugador ha evolucionado. Un 
	ejemplo de este bucle dentro de un juego es un enemigo del jefe, en un principio el 
	jugador no podrá derrotarlo pero después de un tiempo de luchar contra él, el 
	jugador podrá derrotarlo dejando un sensación de progresión en este
	\cite{RefIntroGamificacion}.  
\end{itemize}

\subsection{Tipos de diversión.}
La diversión es uno de los elementos principales obre los que se apoya la 
gamificación para generar experiencias de usuario únicas. En trabajo “Why we 
play games: Four keys to more emotion without story” se propone la existencia 
de cuatro tipos de diversión que los videojuegos utilizan para generar 
experiencias de juego que logran enganchar a sus jugadores:

	\begin{itemize}
		\item \textbf{Diversión difícil}. Este tipo de diversión es la que se da 
		cuando el jugador logra superar retos dentro del juego. Los jugadores disfrutan 
		de este tipo de diversión cunado pueden ver que tan buenos son en el juego, 
		superan múltiples objetivos o requieren más de estrategia para ganar en lugar 
		de suerte. Las emociones que despierta este tipo de diversión es de éxito o 
		frustración\cite{RefDiversion}. 
	
		\item \textbf{Diversión fácil}. Este tipo de diversión viene de descubrir el 
		mundo del juego. Para lograr esto el juego se vale de elementos que despierten 
		la curiosidad del jugador. El jugador disfruta de este tipo de diversión cuando 
		explora nuevos mundos, se siente uno con el personaje, logra descubre misterios 
		del mundo o de la narrativa del juego\cite{RefDiversion}.  
	
		\item \textbf{Diversión seria (estados alterados)}. Esta diversion se vale de 
		como la partida puede afectar el estado de animo del jugador. Esta diversión se 
		logra cuando: El jugador evita el aburrimiento, se siente mejor consigo mismo, 
		aclara su mente al completar un nivel. En este tipo de diversión el juego pasa 
		a ser un factor de desahogo personal\cite{RefDiversion}.
	
	
		\item \textbf{El factor persona}. Con este tipo de diversión el juego se 
		convierte en un medio de convivencia con otras personas en donde la interacción 
		con otros jugadores es el máximo factor de disfrute y se encuentra por sobre 
		el juego\cite{RefDiversion}.
	\end{itemize} 

 Dado que la gamificación toma elementos del videojuego y los pone en entornos ajenos 
 a éste, estos cuatro tipos de diversión también pueden ser empleados en el diseño 
 de experiencias de gamificación. 


\subsection{Framework de diseño de la Gamificación}
A continuación se presenta el framework de diseño creado por el profesor Kevin 
Werbach de la Universidad de Pensilvania. El framework del profesor Werbach está 
compuesto por seis pasos:

\begin{itemize}
	\item \textbf{Definir el objetivo del negocio}. Este paso consiste en definir 
	el objetivo que persigue la gamificación, no se debe de confundir con el 
	comportamiento objetivo. En este punto de contesta la pregunta ¿Cómo beneficiará 
	la implementación de la gamificación al proyecto?
	\item \textbf{Definir el comportamiento objetivo}. Aqui se define el 
	comportamiento que se quiere modificar o inculcar al usuario para que se cumpla 
	el objetivo del negocio. Este comportamiento debe quedar definido en cualidades 
	cuantificables que permitan su verificación.
	\item \textbf{Describir a los jugadores}. Una aplicación exitosa no se enfoca 
	en un solo tipo de jugador sino logra incorporar dinamicas atractivas para los 
	cuatro tipos de jugadores.
	\item \textbf{Diseñar los intervalos de actividades}. En este paso se 
	intercalan los lapsos de compromiso de la aplicación y los lapsos de progresion 
	del comportamiento. Es importante que se realice una buena implementación de 
	dichos intervalos, pues de esto depende que el usuario utilice o no la 
	aplicación por largos periodos de tiempo.
	\item \textbf{No olvidar la diversión}\cite{RefIntroGamificacion}. 
\end{itemize}
